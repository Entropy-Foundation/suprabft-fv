\documentclass{beamer}
\usepackage{amsmath}
\usepackage{graphicx}
\mode<presentation>
{
  \usetheme{default}
  \usecolortheme{seahorse}
}

\title{Formal Verification of Moonshot SMR Protocol}
\author{ M.~Praveen, Chandradeep Dey, Namratha Reddy
\\ Acknowledging: Isaac Doidge, Raghavendra Ramesh}
\institute{Chennai Mathematical Institute ~\textbar~ Supra Oracles }
\date{November 2023}

\begin{document}
\begin{frame}
    \titlepage
\end{frame}

\begin{frame}
    \frametitle{Goal of the Project}
    \begin{itemize}
        \item Supra Research has invented highly efficient (both in
            terms of throughput and latency) Chained Moonshot State
            Machine Replication protocol. White paper is available on
            Supra website and the academic version will be soon on
            ePrint.
            \pause
            \vfill
        \item The papers contain full safety and liveness proofs, hand
            written by experts.
            \pause
            \vfill
        \item The safety and liveness proofs and written in English
            language, prone to errors and overlooks by humans. Even
            proofs in classical mathematics published in peer reviewed
            journals are known to have errors (subtle, hard to
            detect).
        \item The goal of this project is to write the above safety
            proof in a formal logical language and verify its
            correctness
            \pause
            \vfill
        \item What is gained from this exercise? \pause
            \vfill
            \begin{itemize}
                \item Confidence that there are no errors in the proof
                    \pause
                    \vfill
                \item Possibility of generating test cases from the
                    formal proof constructed
            \end{itemize}
    \end{itemize}
\end{frame}

\begin{frame}
    \frametitle{Tools and Methods Used}
    \begin{itemize}
        \item \alert{IVy}: a deductive
            verification tool, verified Tendermint, HotStuff etc.
            \pause
            \vfill
        \item Moonshot pseudocode $\Rightarrow$ IVy model
            \pause
            \vfill
        \item Desired safety property $\Rightarrow$ IVy invariants
            \pause
            \vfill
        \item Does the model violate the invariants? $\Rightarrow$ Is a set of
            logical formulas satisfiable? \pause $\Rightarrow$ Microsoft Z3
            \pause
            \vfill
        \item Is the set of logical formulas satisfiable? \alert{Yes}:
            an example execution violating the invariant
            \pause
            \vfill
        \item Is the set of logical formulas satisfiable? \alert{No}:
            protocol is safe
            \pause
            \vfill
        \item Main challenge: Z3 can get stuck for hours without
            saying yes or no. \pause Need to know logic to overcome
            this
    \end{itemize}
\end{frame}

\begin{frame}
    \frametitle{Artefacts of the project}
     https://github.com/Entropy-Foundation/suprabft-fv/tree/master/suprabft    

     \alert{moonshot.ivy} - Moonshot pseudocode written in IVy
     language\\
     \includegraphics[scale=0.25]{OptimisticPropProc.png}
\end{frame}

\begin{frame}
    \frametitle{Artefacts of the project}
    \alert{quorum\textunderscore{}verification.ivy} - IVy shortcut
    for cryptographic checks of quorum certificates
    \includegraphics[scale=0.25]{QuorumDef.png}
\end{frame}

\begin{frame}
    \frametitle{Artefacts of the project}
    \alert{safety.ivy} - properties to be verified
    \includegraphics[scale=0.25]{FullSafety.png}
\end{frame}

\begin{frame}
    \frametitle{Byzantine nodes}
    They can send any kind of message to anybody, posing as any other
    Byzantine node
    \includegraphics[scale=0.25]{ByzantineSendMsg.png}
    \includegraphics[scale=0.25]{RecordByzantineAction.png}
\end{frame}

\begin{frame}
    \frametitle{Every Minute Detail is Checked}
    Example - If an honest processor votes for a block B, B's parent
    Bp has a lesser round
    \includegraphics[scale=0.25]{ParentBlockEarlier.png}
    \pause
    \includegraphics[scale=0.25]{ParentBlockEarlierPass.png}
\end{frame}

\begin{frame}
    \frametitle{Every Minute Detail is Checked}
    Depends on another property: while processing fallback proposals,
    all the QC's in the TC are processed first, so that the current
    round 
    r\textunderscore{}c is strictly greater
    than maxQC of the TC 
    \includegraphics[scale=0.25]{RcGtMaxQC.png}
    \pause
    Suppose we omit the other property
    \includegraphics[scale=0.25]{RcGtMaxQCOmit.png}
    \pause
    IVy gives a counter example - block B of round 1 can have parent
    block Bp of round 2
    \includegraphics[scale=0.25]{ParentBlockEarlierViolated.png}
\end{frame}

\begin{frame}
    \frametitle{Structure of the safety proof}
    Follows the structure of the proof from Supra Research teams'
    paper

    Theorem 1 is the full safety, the last invariant in safety.ivy
    \includegraphics[scale=0.25]{FullSafety.png}
\end{frame}

\begin{frame}
    \frametitle{Structure of the safety proof}

    Depends on Corollary 4, called
    latest\textunderscore{}committed\textunderscore{}ancestors in
    safety.ivy
    \includegraphics[scale=0.25]{Corollary4.png}

    Proving Corollary 4 is laborious, depends on many other properties
\end{frame}

\begin{frame}
    \frametitle{Structure of the safety proof}
    \begin{itemize}
        \item For proving one invariant, we may need to prove another
            one first \dots
            \pause
            \vfill
        \item Some invariant may be valid, but Z3 gets stuck, so we
            have to logically decompose it into smaller ones for Z3 to
            handle \dots
            \pause
            \vfill
        \item There are totally around 250 invariants that finally
            prove Theorem 1. All are manually written
            \pause
            \vfill
        \item Moonshot has a complicated pipelining mechanism to
            improve communication complexity. This is the first time a
            such a complicated protocol has been formally verified
    \end{itemize}
    \pause
    \vfill
    \begin{center}
        \Large{Thank You}
    \end{center}
\end{frame}
\end{document}

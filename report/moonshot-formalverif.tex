\documentclass[a4paper,UKenglish,cleveref, autoref, thm-restate]{oasics-v2021}

\pdfoutput=1
\bibliographystyle{plainurl}
\usepackage{amsmath}
\usepackage{amsfonts}
\usepackage{amssymb}
\bibliographystyle{plainurl}
\title{Formal Verification Chained Moonshot Byzantine Fault Tolerant State Machine Replication Protocol}
\titlerunning{Formal Verification of Chained Moonshot SMR Protocol}
\author{M.~Praveen}{Chennai Mathematical Institute \and ReLaX}{}{}{Partially funded by a grant from the Infosys foundation}
\author{Issac Doidge}{SupraOracles}{}{}{}
\author{Raghavendra Ramesh}{SupraOracles}{}{}{}
\authorrunning{M.~Praveen, Issac Doidge, Raghavendra Ramesh}
\Copyright{M.~Praveen, Issac Doidge, Raghavendra Ramesh}
\ccsdesc[500]{Networks~Protocol testing and verification}
\ccsdesc[500]{Theory of computation~Logic and verification}
\ccsdesc[500]{Theory of computation~Automated reasoning}
\begin{CCSXML}
	<ccs2012>
	<concept>
	<concept_id>10003033.10003039.10003041.10003042</concept_id>
	<concept_desc>Networks~Protocol testing and verification</concept_desc>
	<concept_significance>500</concept_significance>
	</concept>
	<concept>
	<concept_id>10003752.10003790.10002990</concept_id>
	<concept_desc>Theory of computation~Logic and verification</concept_desc>
	<concept_significance>500</concept_significance>
	</concept>
	<concept>
	<concept_id>10003752.10003790.10003794</concept_id>
	<concept_desc>Theory of computation~Automated reasoning</concept_desc>
	<concept_significance>500</concept_significance>
	</concept>
	</ccs2012>
\end{CCSXML}
\keywords{Distributed consensus, Formal verification}
\acknowledgements{We would like to thank Chandradeep Dey and Namrata Reddy, who were part of this project during its initial phase.}
%\nolinenumbers %uncomment to disable line numbering

%Editor-only macros:: begin (do not touch as author)%%%%%%%%%%%%%%%%%%%%%%%%%%%%%%%%%%
\EventEditors{John Q. Open and Joan R. Access}
\EventNoEds{2}
\EventLongTitle{42nd Conference on Very Important Topics (CVIT 2016)}
\EventShortTitle{CVIT 2016}
\EventAcronym{CVIT}
\EventYear{2016}
\EventDate{December 24--27, 2016}
\EventLocation{Little Whinging, United Kingdom}
\EventLogo{}
\SeriesVolume{42}
\ArticleNo{23}
%%%%%%%%%%%%%%%%%%%%%%%%%%%%%%%%%%%%%%%%%%%%%%%%%%%%%%
\begin{document}

	\maketitle
	\begin{abstract}
Decentralized Finance (DeFi) have emerged as a contemporary competitive as well as complementary to traditional centralized finance systems. As of 23rd January 2024, per Defillama approximately USD 55 billion is the total value locked on the DeFi applications on all blockchains put together.

A Byzantine Fault Tolerant (BFT) State Machine Replication (SMR) protocol, popularly known as consensus protocol, is the central component of a blockchain. Forks in blockchains imply possible double spending attacks and are catastrophic given high volumes of finance that are transacted on blockchains. Formal verification of the safety of BFT SMR protocols provide the highest guarantee possible but is considered as complex and challenging. This is reflected by the fact that not many complex consensus protocols are formally verified except for Tendermint. 

We focus on Chained Moonshot consensus protocol. Similar to Tendermint's formal verification we too model Chained Moonshot using IVy and formally prove that for all network sizes, as long as the number of Byzantine validators is less than the 1/3, the protocol never forks, thus proving that double spending is not possible.
	\end{abstract}
\newcommand{\prcs}{\mathcal{V}}
\newcommand{\blkchn}{\mathbf{B}}

\section{Introduction}
Public blockchain networks are revolutionising modern society by facilitating decentralised, immutable and verifiable data exchange. At the heart of these networks are consensus protocols that enable state machine replication (SMR). A blockchain network is a form of distributed state machine, which is transitioned from one state to another by applying client-submitted instructions called transactions. SMR protocols ensure that every node in the network maintains a consistent state by facilitating their agreement upon the order in which these transactions should be executed. A Byzantine Fault Tolerant (BFT) SMR protocol is one that tolerates a fixed number of faulty participants. These faulty processes, termed Byzantine, may crash or deviate arbitrarily from the protocol, but are assumed to be unable to break cryptographic primitives like signatures. Blockchain-based SMR protocols group transactions into blocks, with each new block committed by the network referencing the previously committed one as its parent, thus forming the blockchain.

Several such protocols have been proposed [citations], some of them also formally verified [citations]. New ones are being proposed, to improve certains aspects. Two such aspects are block period and commit latency. Block period is the delay between consecutive block proposals. Commit latency is the delay between proposal of a block and achieving $2/3$ majority of processes committing the block. Chained Moonshot protocol achieves better block period and commit latency compared to other similar protocols [citations].

One of the main desirable properties of such protocols is \emph{safety}: any two honest processes must agree on the set of transactions executed and the order in which they are executed. Formally, if two honest processes have committed chains of blocks, then one of the chains must be (not necessarily strict) prefix of the other one. Chained Moonshot protocol is proved to be safe and live [citation], with a handwritten proof. The goal of this project is to give a proof of safety in a formal verification tool. The usage of such protocols in commercial settings makes mistakes in their design costly. On the other hand, distributed systems are inherently complex and trying to squeeze as little block period and commit latency as possible makes them even more complicated. Designing protocols and proving them correct by hand are notoriously prone to errors; see [Holistic Verification of Blockchain Consensus, DISC 2022] and references therein.

We chose a tool called IVy for this project. Ivy is a language and a tool for the formal specification and verification of distributed systems. Ivy supports deductive verification using automated provers, model checking, automated testing, manual theorem proving and generation of executable code. In order to achieve greater verification productivity, a key design goal for Ivy is to allow the engineer to apply automated provers in the realm in which their performance is relatively predictable, stable and transparent. In particular Ivy focuses on the use of decidable fragments of first-order logic.

\paragraph{Our Contribution}
\begin{itemize}
	\item We formally model the chained Moonshot protocol in Ivy.
	\item Within Ivy, we formally verify that the proof of safety as given in [citation] is correct.
	\item Within Ivy, we explicitly state and prove many facts that are used implicitly in the handwritten safety proof.
	\item For proving properties in Ivy, we need to manually write invariants of the protocol that are inductive and also imply the desired safety property that we wish to prove. We identify several properties the conjunction of which is proved to be inductive by Ivy and also imply the desired safety property.
\end{itemize}

\paragraph{Related Work} Formal verification of distributed consensus protocols have been done before. The safety and liveness of the Tendermint protocol has been formally verified [citation] in Ivy. A simplified version of HotStuff protocol is formally verified to be safe in Ivy as well as TLA, a formal specification and verification tool. Multi-Paxos protocol has been formally verified to be safe in TLA\textsuperscript{+} [FM2016]. There have been many attempts at formally verifying Paxos and its many variants [PaxosMadeEPR, AutomaticProofOfPaxosIC3PO], trying to automate the verification as much as possible. TODO: Ironfleet, Verdi

The above works are based on deductive verification, with the protocol modeled in languages based on a logical language and the properties to be proved are converted to (un)satisfiability of formulas. Another approach is model checking, where the protocol is modeled as a state machine and the desired properties are written in variants of temporal logics. Algorithms then attempt to verify that the state machine satisfies the temporal properties. PSync [citation] and related tools use this approach. The RedBelly block chain consensus protocol has been verified for safety and liveness using this approach [citation].

Even after a protocol has been formally proved to be correct, its actual implementation may be incorrect due to other issues. Very often, protocol specifications use abstractions to describe some aspects of the protocol that may be subject to different refinements by implementers. This process may introduce errors in the program. Aiming to capture such errors, in [citation], the reference implementation of the Ethereum 2.0 Beacon chain has been formally verified.



\section{The Chained Moonshot SMR Protocol}
In this section we give some details of the system we want to formally verify and setup the scope and goal of our experiment.

\subsection{System Description of the Chained Moonshot SMR protocol}
The system under study is a block-chain based SMR protocol, in which client transactions are grouped in blocks that explicitly refer to one another to form a blockchain. A set of processes $\prcs$ participate in the protocol. Some of the processes may be faulty --- they may crash or deviate arbitrarily from the protocol. Such faulty processes are termed \emph{Byzantine}. It is assumed that less than one thirds of the processes are Byzantine and that they are unable to break cryptographic primitives like signatures. It is assumed that each block refers to at most one previously proposed block as its \emph{parent}, and that these blocks are proposed in sequential \emph{rounds} by an elected \emph{leader} process, which may be distinct for different rounds. Every process $v \in \prcs$ maintains a local copy, denoted by $\blkchn_v$, of the \emph{canonical blockchain}.

The formal safety property is that for every run of the protocol, for each pair of honest processes $(v_i,v_j) \in \prcs \times \prcs$, either $\blkchn_{v_i}$ is a (not necessarily strict) prefix of $\blkchn_{v_j}$ or vice-versa.

The chained Moonshot protocol achieves better block period and commit latency via some optimizations, which we explain by first describing the simpler Tendermint protocol [citation]. Tendermint is an adaptation of the Practical Byzantine Fault Tolerant (PBFT) protocol [citation] to the blockchain setting. One instance of the Tendermint protocol aims to achieve consensus for appending one block to the canonical block chain. Each instance proceeds in three phases: \emph{Propose, Prepare} and \emph{Commit}. In the first phase, the leader of the current round multicasts a block in a signed Proposal message. A validator then enters the Prepare phase after receiving a valid Proposal from the leader, and re-multicasts it along with a Prepare message to indicate its endorsement of the Proposal. The validator then waits to receive a quorum of valid Prepare messages and then constructs a Prepare Quorum Certificate (Prepare QC) as verifiable proof that a quorum of processes have accepted the leader’s proposal. After forming this QC, the process enters the Commit phase and multicasts a Commit message as a second endorsement of the Proposal. As before, the process then waits to receive a quorum of valid Commit messages before forming Commit Quorum Certificate and committing the block.

Chained HotStuff [citation]  introduced the concept of a message serving multiple purposes, allowing leaders to create new proposals justified by the prepare QC of the proposal of their predecessor. The Chained Moonshot protocol takes this observation one step further, wherein the Prepare phase of an earlier round can be safely overlapped with the Propose phases of later ones. Some proposals may fail due to various reasons and the parent of a block need not be the one proposed in the previous round, but several rounds before. Some parts of the safety proof thus proceeds by induction on number of rounds.  Some rules of the protocol refer to ancestors of a block. The ancestor relation is the transitive closure of the parent relation. Neither induction nor transitive closure are definable in first-order logic. These are some of the challenges we faced, which we will come back to later.

\subsection{Ivy verification tool}
Ivy is a language and a tool for the formal specification and verification of distributed systems. Systems are represented as state transition machines. States are multi-sorted first-order structures, with relations and functions. Transitions specify how the state is muted. Any update definable in first-order logic is supported. Update instructions can be given in sequence one after another, giving the syntax the flavor of an imperative programming language. Multiple update instructions can be grouped together into an \emph{action}, a keyword in Ivy used to denote state transition specifications.

The system under consideration can be split into multiple modules, with internal states of a module not allowed to be modified directly by other modules. One module can call actions of another module, passing parameters. Modules can reason about one another using assume gurantee specifications, which are formulas specifying properties of the modules' states. Properties of the overall system has to be proved by writing \emph{inductive invariants}, which are properties satisfying two conditions --- initiation and inductiveness. Initiation means that the initial state of the system satisfies the invariant. Inductiveness means that if any of the actions are executed in any state that satisfies the invariant, the resulting state also satisfies the invariant.

\subsection{Objectives and Scope of the Experiment}
The objective of this project is to formally verify the safety of chained Moonshot protocol. Formal verification ensures that there are no errors in the reasoning underlying the handwritten proof. The development and proof of safety and liveness of the chained Moonshot protocol underwent many cycles (some modificatins to ensure liveness and some for simplifying the specification and proofs). The process of formally verifying safety uncovered some points in the specifications and proofs that were ambiguous and helped better understand many details that were implicit in the handwritten proofs.

Only a high level abstract specification of the protocol is modeled and verified. Some implementation details are hence modeled with Boolean abstractions. For example, timers used in the protocol are replaced by Boolean propositions that indicate whether or not a timer has expired. In the Ivy model, the Boolean proposition can switch value anytime non-deterministically to simulate a timer getting expired or reset, instead of tracking the actual time elapsed since the last reset. This is a sound abstraction for proving safety.

Another abstraction we have adapted from the literature is handling quorums. The protocol specification mandates that a process needs to receive messages from two thirds majority of all processes in order to achieve a quorum. Verifying this detail would require having arithmetic in the formulas passed on to SMT solvers, potentially affecting the solvers' performance. Instead, what is modeled is the \emph{quorum intersection property} [citation] --- any two quorums have at least one common honest process. It is this property of quorums that are mainly used in correctness proofs and is modeled in Ivy as an axiom, avoiding the usage of arithmetic.

Processes receive messages from the network and verify their authenticity by checking digital signatures. It is assumed that Byzantine processes cannot break cryptographic primitives and hence they cannot imitate signatures of honest processes. Checking digital signatures is not modeled in Ivy --- the model assumes messages sent by honest processes are authentic. The model also disallows byzantine processes to send messages on behalf of other honest process, though they can send any kind of message on behalf of other byzantine processes, even if such a message is not mandated to be sent by the protocol specification.
\end{document}